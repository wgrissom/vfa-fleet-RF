\documentclass[11pt]{article}
% Command line: pdflatex -shell-escape LT_Pulse.tex
\usepackage{geometry} 
\usepackage{amsmath}
\usepackage{amssymb}
\usepackage{booktabs}
\usepackage{times}
\usepackage{bm}
\usepackage{fixltx2e}
\usepackage[outerbars]{changebar}
\usepackage{graphicx}
\usepackage{epstopdf}
\usepackage{color}
\usepackage{tabularx}

\epstopdfsetup{suffix=} % to remove 'eps-to-pdf' suffix from converted images
%\usepackage{todonotes} % use option disable to hide all comments

\usepackage[sort&compress]{natbib}
\bibpunct{[}{]}{,}{n}{,}{,}

\usepackage[noend]{algpseudocode}

\usepackage{dsfont}
\usepackage{relsize}
\usepackage{gensymb}

%changing the Eq. tag to use [] when numbering. use \eqref{label} to reference equations in text.
\makeatletter
  \def\tagform@#1{\maketag@@@{[#1]\@@italiccorr}}
\makeatother

\linespread{1}
%\setlength{\parindent}{0in}

% the following command can be used to mark changes made due to reviewers' concerns. Use as \revbox{Rx.y} for reviewer x, concern y.
\newcommand{\revbox}[1]{\marginpar{\framebox{#1}}}
%\newcommand{\revbox}[1]{}

%\newcommand{\bop}{$\vert B_1^+ \vert$}
\newcommand{\kt}{$k_\textrm{T}$}
\newcommand{\bone}{$B_1^+$}
\newcommand{\bmap}{$B_1^+$}
\newcommand{\mytilde}{\raise.17ex\hbox{$\scriptstyle\mathtt{\sim}$}}  % tilde symbol
\mathchardef\mhyphen="2D

\begin{document}



\newpage

\title{Shinnar-Le Roux RF Pulse Design for Variable Flip Angle FLEET: Theory}
\author{Will Grissom}
\maketitle
\begin{flushleft}
Vanderbilt University Institute of Imaging Science \\
Department of Biomedical Engineering \\
Vanderbilt University, Nashville, TN, United States \\
\end{flushleft}
\thispagestyle{plain}

\pagebreak

In variable flip angle (VFA) FLEET acquisitions, 
interleaved segments of EPI readouts are acquired consecutively, 
with no delay between them. 
This reduces the likelihood of phase shifts and motion between segments which otherwise 
cause ghosting and motion artifacts in conventional segmented acquisitions. 
Because of the short recovery times between RF pulses in VFA-FLEET,
there is negligible time for longitudinal relaxation, 
and the longitudinal magnetization available immediately before each RF pulse is a function of both 
the previous RF pulse's rotation parameters
and the longitudinal magnetization available immediately before the previous RF pulse.
Thus, the longitudinal magnetization profile changes between excitations and must be accounted for in the RF pulse design in order
to maintain a consistent transverse magnetization (signal) profile across the imaged slice. 
In this document, I describe a strategy to recursively design slice-selective RF pulses for FLEET acquisitions using the Shinnar-Le Roux (SLR) algorithm.

\par There are two boundary conditions for this problem. 
Because longitudinal magnetization is replenished during the long delay between the last segment of one acquisition and the first segment of the next, 
we assume $M_z = 1$ before the first pulse, 
and the last pulse should have a 90-degree flip angle to maximize signal by completely depleting the remaining longitudinal magnetization.
Since the longitudinal magnetization before the last pulse is unknown, 
we must start by designing the first pulse.
However, designing this pulse requires knowledge of its nominal passband flip angle,
which can be calculated recursively starting with the last flip angle as:
\begin{equation}
\theta_{i-1} = \tan^{-1} \sin \theta_i,
\label{eq:fliprecursion}
\end{equation}
where $i$ indexes segments.
Given the first pulse's flip angle, the conventional SLR algorithm can be used to calculate the first pulse. 
That pulse will have rotation parameters $A_1$ and $B_1$, which are functions of frequency.
Given those parameters, the longitudinal magnetization before the second segment's pulse is:
\begin{equation}
M_z^{i+1} = M_z^i \left( 1 - 2 \vert B_i \vert^2\right). 
\label{eq:mzrecursion}
\end{equation}
To solve for the second segment's pulse, 
we need its complex-valued $B$ profile. 
First we solve for its magnitude, 
using the fact that the magnitude of the transverse magnetization it excites should be the same as the first pulse's magnitude,
as:
\begin{equation}
\vert M_{xy}^1 \vert = \vert M_{xy}^{i+1} \vert =  \vert M_z^{i+1} 2 A^*_{i+1} B_{i+1} \vert = \vert M_z^{i+1} \vert 2 \sqrt{1 - \vert B_{i+1} \vert^2} \vert B_{i+1} \vert,
\end{equation}
where the last equality results from the fact that $\vert A \vert^2 + \vert B \vert^2 = 1$.
This equation is quadratic in $\vert B_{i+1} \vert^2$, so the quadratic equation can be applied to solve for $\vert B_{i+1} \vert$.
Then the complex-valued $A_{i+1}$ can be solved by assuming a min-power pulse [Pauly SLR],
and the phase of $B_{i+1}$ can be obtained from $A_{i+1}$ and $M_{xy}^1$ as:
\begin{equation}
\angle B_{i+1} = \angle \left(\frac{M_{xy}^1}{M_z^{i+1} 2 A_{i+1}^* }\right).
\label{eq:bphase}
\end{equation}
Then the RF pulse for segment $i+1$ can be calculated from the complex-valued $A$ and $B$ profiles using the inverse SLR transform. 

\par It is possible to take into account the effect of longitudinal relaxation in the pulse designs, 
by modifying the $M_z$ equation as:
\begin{equation}
M_z^{i+1} = M_z^i \left( 1 - 2 \vert B_i \vert^2\right)e^{-TR_{seg}/T_1} + M_0\left(1 - e^{-TR_{seg}/T_1}\right),
\label{eq:MzT1}
\end{equation}
where $TR_{seg}$ is the time between excitation of each segment. 
Note that we should also adjust the first flip angle to account for longitudinal relaxation in the sequence,
or else we may not use all the available signal. 
This is complicated by the fact that Eq. \ref{eq:fliprecursion} is derived by setting consecutively excited transverse magnetizations equal to each
other, as:
\begin{align}
M_{xy}^{i} &= M_{xy}^{i+1} \\
M_z^{i}\sin \theta_{i} &= M_z^{i+1}\sin \theta_{i+1} \\
				&= \left(M_z^{i}\cos \theta_{i} e^{-TR_{seg}/T_1} + M_0\left(1-e^{-TR_{seg}/T_1}\right)\right) \sin \theta_{i+1}.
\end{align}
The recursion in Eq. \ref{eq:fliprecursion} is obtained by assuming $T_1 \approx \infty$, 
in which case the unknown $M_z^i$ divides out of the above equation.
If a finite $T_1$ is assumed, this is no longer the case and a more sophisticated solution is needed.
For now, we are ignoring this effect since it should only result in a slightly reduced signal, 
but the slice profiles will still be maintained if we include the modification of Eq. \ref{eq:MzT1} in the design. 

\par A recursive design can also be established for spin echo sequences, 
where the effect of the refocusing pulse on the longitudinal magnetization must be taken into account.
The combined rotation of a general excitation and refocusing pulse pair with crushing around the refocusing pulse is given by:
\begin{align}
\left(\begin{array}{c} A^{se} \\ B^{se} \end{array}\right) &= \left(\begin{array}{cc} A^{ref}e^{-\imath \phi(x)} & -(B^{ref})^* \\ B^{ref} & (A^{ref})^*e^{+\imath \phi(x)} \end{array}\right) \left(\begin{array}{cc} A^{ex} \\ B^{ex} \end{array}\right) \\
&= \left(\begin{array}{c} A^{ex}A^{ref}e^{-\imath \phi(x)} - (B^{ref})^*B^{ex} \\ A^{ex}B^{ref} + (A^{ref})^* B^{ex} e^{+\imath \phi(x)} \end{array}\right),
\end{align}
where $\phi(x)$ is the phase induced by the crusher gradients surrounding the refocusing pulse. 
Substituting this expression for $B^{se}$ into Eq. \ref{eq:mzrecursion}, we get:
\begin{align}
M_z^{i+1} &= M_z^i \left( 1 - 2 \left( \vert A^{ex} B^{ref} \vert^2 + \vert (A^{ref})^* B^{ex} \vert^2 + 2\Re \left\{ A^{ex} B^{ref} (A^{ref})^* B^{ex} e^{\imath \phi(x)} \right\} \right) \right) \\
&\approx M_z^i \left(1 - 2\vert A^{ex} B^{ref} \vert^2\right),
\end{align}
where the approximation comes from the fact that $A^{ref} \approx \cos\left(\pi/2\right) = 0$, 
and that $e^{\imath \phi(x)}$ integrates to zero through the slice. 
Unlike the gradient echo case,
because both $A^{ex}$ and $B^{ref}$ have magnitudes close to 1, 
the sign of $M_z$ will flip between each segment. 
This will be coupled into the design via Eq. \ref{eq:bphase}.


\end{document}
